%%%%%%%%%%%%%%%%%%%%%%%%%%%%%%%%%%%%%%%%%%%%%%%%%%%%%%%%%%%%%%%%%%%%%%%%%%%%%%%%%%%%%%%%%%%%%%%%
%
% CSCI 1430 Written Question Template
%
% This is a LaTeX document. LaTeX is a markup language for producing documents.
% Your task is to answer the questions by filling out this document, then to
% compile this into a PDF document.
%
% TO COMPILE:
% > pdflatex thisfile.tex
%
% If you do not have LaTeX and need a LaTeX distribution:
% - Departmental machines have one installed.
% - Personal laptops (all common OS): http://www.latex-project.org/get/
%
% If you need help with LaTeX, come to office hours. Or, there is plenty of help online:
% https://en.wikibooks.org/wiki/LaTeX
%
% Good luck!
% James and the 1430 staff
%
%%%%%%%%%%%%%%%%%%%%%%%%%%%%%%%%%%%%%%%%%%%%%%%%%%%%%%%%%%%%%%%%%%%%%%%%%%%%%%%%%%%%%%%%%%%%%%%%
%
% How to include two graphics on the same line:
%
% \includegraphics[width=0.49\linewidth]{yourgraphic1.png}
% \includegraphics[width=0.49\linewidth]{yourgraphic2.png}
%
% How to include equations:
%
% \begin{equation}
% y = mx+c
% \end{equation}
%
%%%%%%%%%%%%%%%%%%%%%%%%%%%%%%%%%%%%%%%%%%%%%%%%%%%%%%%%%%%%%%%%%%%%%%%%%%%%%%%%%%%%%%%%%%%%%%%%

\documentclass[11pt]{article}

\usepackage[english]{babel}
\usepackage[utf8]{inputenc}
\usepackage[colorlinks = true,
            linkcolor = blue,
            urlcolor  = blue]{hyperref}
\usepackage[a4paper,margin=1.5in]{geometry}
\usepackage{stackengine,graphicx}
\usepackage{fancyhdr}
\setlength{\headheight}{15pt}
\usepackage{microtype}
\usepackage{times}

% python code format: https://github.com/olivierverdier/python-latex-highlighting
\usepackage{pythonhighlight}

\frenchspacing
\setlength{\parindent}{0cm} % Default is 15pt.
\setlength{\parskip}{0.3cm plus1mm minus1mm}

\pagestyle{fancy}
\fancyhf{}
\lhead{Project 3 Questions}
\rhead{CSCI 1430}
\rfoot{\thepage}

\date{}

\title{\vspace{-1cm}Project 3 Questions}


\begin{document}
\maketitle
\vspace{-2cm}
\thispagestyle{fancy}

\section*{Instructions}
\begin{itemize}
  \item 5 questions.
  \item Write code where appropriate.
  \item Feel free to include images or equations.
  \item Please make this document anonymous.
  \item On upload, \textbf{Gradescope will ask you to assign question numbers to your pages}. Making each question end with a page break after your answer is a good way to ease this process. \textbf{Failing to assign page numbers will result in a deduction.}
\end{itemize}

\section*{Questions}

%%%%%%%%%%%%%%%%%%%%%%%%%%%%%%%%%%%

\paragraph{Q1:} In machine learning, what are bias and variance? When we evaluate a classifier, what are overfitting and underfitting, and how do these relate to bias and variance?

%%%%%%%%%%%%%%%%%%%%%%%%%%%%%%%%%%%
\paragraph{A1:} Your answer here.



%%%%%%%%%%%%%%%%%%%%%%%%%%%%%%%%%%%

\pagebreak
\paragraph{Q2:} Given a linear classifier like SVM, how might we handle data that are not linearly separable? How does the \emph{kernel trick} help in these cases? (See hidden slides in supervised learning crash course deck, plus your own research.)

%%%%%%%%%%%%%%%%%%%%%%%%%%%%%%%%%%%
\paragraph{A2:} Your answer here.



\pagebreak
\paragraph{Q3:} Given a linear classifier such as SVM which separates two classes (binary decision), how might we use multiple linear classifiers to create a new classifier which separates $k$ classes?

Below, we provide pseudocode for a linear classifier. It trains a model on a training set, and then classifies a new test example into one of two classes. Please convert this into a multi-class classifier. You can take either the one vs.~all (or one vs.~others) approach or the one vs.~one approach in the slides; please declare which approach you take.

\emph{Hints:} Be aware that 1) the input labels in the multi-class case are different, and you will need to match the expected label input for the \texttt{train\_linear\_classifier} function, 2) you need to make a new decision on how to aggregate or decide on the most confident prediction.

\emph{Note:} A more efficient software application would separate the classifier training and testing into two different functions so that the model could be reused without retraining.

%%%%%%%%%%%%%%%%%%%%%%%%%%%%%%%%%%%
\paragraph{A3:} Your answer here.

\begin{python}
# Inputs
#   train_feats: N x d matrix of N features each d descriptor long
#   train_labels: N x 1 array containing values of either -1 (class 0) or 1 (class 1)
#   test_feat: 1 x d image for which we wish to predict a label
#
# Outputs
#   -1 (class 0) or 1 (class 1)
#
# Please turn this into a multi-class classifier for k classes.
# Inputs:
#    As before, except
#    train_labels: N x 1 array of class label integers from 0 to k-1
# Outputs:
#    A class label integer from 0 to k-1
#
def classify(train_feats, train_labels, test_feat)
    # Train classification hyperplane
    weights, bias = train_linear_classifier(train_feats, train_label)
    # Compute distance from hyperplane
    test_score = weights * test_feats + bias

    return 1 if test_score > 0 else -1
\end{python}


%%%%%%%%%%%%%%%%%%%%%%%%%%%%%%%%%%%

\pagebreak
\paragraph{Q4:} Suppose we are creating a visual word dictionary using SIFT and k-means clustering for a scene recognition algorithm. Examining the SIFT features generated from our training database, we see that many are almost equidistant from two or more visual words. Why might this affect classification accuracy?

Given the situation, describe \emph{two} methods to improve classification accuracy, and explain why they would help.

%%%%%%%%%%%%%%%%%%%%%%%%%%%%%%%%%%%
\paragraph{A4:} Your answer here.



%%%%%%%%%%%%%%%%%%%%%%%%%%%%%%%%%%%

\pagebreak
\paragraph{Q5:} The way that the bag of words representation handles the spatial layout of visual information can be both an advantage and a disadvantage. Describe an example scenario for each of these cases, plus describe a modification or additional algorithm which can overcome the disadvantage.

How might we evaluate whether bag of words is a good model?

%%%%%%%%%%%%%%%%%%%%%%%%%%%%%%%%%%%
\paragraph{A5:} Your answer here.




%%%%%%%%%%%%%%%%%%%%%%%%%%%%%%%%%%%

\end{document}
